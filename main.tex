\documentclass[14pt, b5paper, sourcefont]{qyxf-report-large}

\usetikzlibrary{shapes,arrows,chains}
\usepackage{xcolor}
\usepackage{fontspec}
\usepackage{pythonhighlight}
\usepackage{sourcecodepro}
\usepackage[framed, numbered]{matlab-prettifier}


\author{magicwenli}
\date{\today}
\class{can't tell}
\id{same}
\title{计算机组成与设计实验}
\subtitle{实验1 基本门电路的设计}

\begin{document}

\maketitle
\tableofcontents


\section{实验目的}
\begin{enumerate}
	\item 掌握Verilog语言框架,编程及调试的方法;
	\item 熟悉Verilog的基本语法;
	\item 掌握iverilog开发平台。
\end{enumerate}

\section{实验内容}
\begin{enumerate}
	\item 利用赋值语句完成一个2输入门电路模块的设计。
	\item 利用赋值语句完成多个(4个以上)门电路之间的级联,形成一个完整的电路。
	\item 在iverilog中完成一个工程的设计、编辑、综合和实现的全过程。
	\item 掌握以上电路的程序结构和风格。
	\item 观察和分析仿真波形,注重输入输出之间的时序关系。
\end{enumerate}

\section{实验要求}
\begin{enumerate}
	\item 画出模块的电路图。
	\item 分析电路的仿真波形。
	\item 记录设计和调试过程。
\end{enumerate}

\section{实验代码及结果}

\subsection{2输入门电路模块的设计}
\subsubsection{模块的电路图}
\subsubsection{分析电路的仿真波形}
\subsubsection{实验设计和调试过程}

\subsection{多个(4个以上)门电路之间的级联}
\subsubsection{模块的电路图}
\subsubsection{分析电路的仿真波形}
\subsubsection{实验设计和调试过程}

\section{调试和心得体会}
\begin{lstlisting}[language=c]
#include <sys/types.h>
#include <stdio.h>
#include <unistd.h>

int main()
{
  pid_t pid,pid1;
  pid=fork();
  if (pid<0)
  {
    fprintf(stderr,"fork fail");
    return 1;
  }
  else if (pid==0)
  {
      pid1=getpid();
      printf("child:pid=%d",pid);      //A
      printf("child:pid1=%d",pid1);    //B
   }
   else
   {
      pid1=getpid();
      printf("parent:pid=%d",pid);     //C
      printf("parent:pid1=%d",pid1);   //D
      wait(NULL);
   }
  return 0;
}
\end{lstlisting}

\end{document}