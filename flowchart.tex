%\documentclass{ctexart}
%\usepackage{Tikz}
%\usetikzlibrary{shapes,arrows,chains}
%\begin{document}

%\centering
% Set up a new layer for the debugging marks, and make sure it is on
% top

\pgfdeclarelayer{marx}
\pgfsetlayers{main,marx}

\providecommand{\cmark}[2][]{%
  \begin{pgfonlayer}{marx}
    \node [nmark] at (c#2#1) {#2};
  \end{pgfonlayer}{marx}
  } 
\providecommand{\cmark}[2][]{\relax} 
%%%%%%%%%%%%%%%%%%%%%%%%%%%%%%%%%%%%%%%%%%%%%%%%%%%%
\begin{tikzpicture}[%
    >=triangle 60,              % Nice arrows; your taste may be different
    start chain=going below,    % General flow is top-to-bottom
    node distance=6mm and 50mm, % Global setup of box spacing
    every join/.style={norm},   % Default linetype for connecting boxes
    ]

\tikzset{
  base/.style={draw, on chain, on grid, align=center, minimum height=4ex},
  proc/.style={base, rectangle, text width=8em},
  test/.style={base, diamond, aspect=2, text width=3.5em},
  term/.style={proc, rounded corners},
  coord/.style={coordinate, on chain, on grid, node distance=6mm and 20mm},
  nmark/.style={draw, cyan, circle, font={\sffamily\bfseries}},
  % -------------------------------------------------
  norm/.style={->, draw},
  free/.style={->, draw},
  cong/.style={->, draw},
  it/.style={font={\small\itshape}}
}
% -------------------------------------------------

\node [term, join]      {上电};
\node [proc, join] (p0) {初始化程序};
\node [proc, join] (p1)  {读取键盘输入};
\node [test, join] (t1) {输入的类型};
\node [proc] (p2) {$0 \sim 9$};
\node [test, join] (t2) {状态类型?};
\node [proc, join] (p3) {新赋值到value1};
\node [test, join] (t3) {是否超过四位?};
\node [proc, join] (p7) {送数码管显示};

\node [proc, right= of p2] (p4) {$+-\times\div$};
\node [proc, join=by free] {置状态数为1};
\node [proc, join=by free] (p5) {清屏};

\node [proc, right= of p4] (p6) {$=$};
\node [proc, join=by free] {运算};
\node [test, join=by free] (t5) {结果是否越位};
\node [proc, join=by free] (p8) {显示越位,并声音报警};

\node [proc, right= of t3] (p9) {报警};
\node [proc, join=by free] (p13) {显示输入无效};
\node [proc, right= of p3] (p10) {新赋值到value2};
\node [proc, right= of t1, yshift=3em] (p11) {清零};
\node [proc, right= of p1, yshift=2em] (p12) {清空内存};
\node [term, right= of p13] (p14) {显示运算结果};
%\node [proc, right= of p13] (p14) {};

% -------------------------------------------------
\node [coord, right=of t1] (c1)  {c1}; 
\node [coord, right =of p4 ] (c2)  {c2}; 
\node [coord, left= of p6] (c0) {c0};
\node [coord, right= of p1] (c3) {};
\node [coord, left= of t5] (c4) {};
\node [coord, below= of p10] (c5) {};
\node [coord, below= of p13] (c6) {};
\node [coord, below= of p7] (c7) {};
\node [coord, right= of t5] (c8) {};
\node [coord, right= of p11] (c9) {};
\node [coord, left= of p1] (c10) {};
\node [coord, right= of t2] (c11) {};


\path (t1.south) to node [near start, xshift=2em] {数字} (p2);
  \draw [o->] (t1.south) -- (p2);
\path (t1.east) to node [near start, yshift=-2em] {结果} (c2); 
  \draw [o->] (t1.east) -| (c2) |- (p6);
 \path (t1.east) to node [near start, yshift=1em] {运算符} (c1); 
   \draw [o->] (t1.east) -- (c1) |- (p4);
 \path (t3.east) to node [near start, yshift=1em] {是} (p9);
  \draw [o->] (t3.east) -- (p9);
  \draw [o->] (t2.east) -- (p10);
  \draw [*->] (p10.south) |- (c5) -| (t3);
  \draw [->] (p11.north) -| (p12.south);
  \draw [*->] (p12.west) -| (c3) |- (p1.east);
  \draw [o->] (t1.east) -- (c1) |- (p11);
  
  \path (t5.west) to node [near start, yshift=1em] {否} (c4);  
  \path (t2.east) to node [near start, yshift=1em] {1} (c11);  
  \draw [o->] (t5.west) -| (c4) |- (p14);
  \draw [*->] (p13.south) -| (c6) -| (c8) |- (c9) |- (p11.east);
  \draw [*->] (p7.south) -- (c7) -| (c10) -- (p1.west);
\end{tikzpicture}
%\end{document}